% Artificial Intelligence Agent Contributors
% Drawn by Alfonso R. Reyes
% November 2020
% Based on "Artificial Intelligence: A Modern Approach" by Peter Norvig and Stuart Russell
% Original diagram: https://github.com/FriendlyUser/LatexDiagrams
\documentclass[border=5pt]{standalone}
\usepackage{xcolor}

	\definecolor{ocre}{HTML}{800000}
	\definecolor{sky}{HTML}{C6D9F1}
	\definecolor{skybox}{HTML}{5F86B3}

\usepackage{tikz}
\usepackage{pgfmath}
\usetikzlibrary{decorations.text, arrows.meta,calc,shadows.blur,shadings}

\renewcommand*\familydefault{\sfdefault} % Set font to serif family

% arctext from Andrew code with modifications:
%Variables: 1: ID, 2:Style 3:box height 4: Radious 5:start-angl 6:end-angl 7:text {format along path} 
\def\arctext[#1][#2][#3](#4)(#5)(#6)#7{

\draw[#2] (#5:#4cm+#3) coordinate (above #1) arc (#5:#6:#4cm+#3)
             -- (#6:#4) coordinate (right #1) -- (#6:#4cm-#3) coordinate (below right #1) arc (#6:#5:#4cm-#3) coordinate (below #1)
             -- (#5:#4) coordinate (left #1) -- cycle;
            \def\a#1{#4cm+#3}
            \def\b#1{#4cm-#3}
\path[
    decoration={
        raise = -0.5ex, % Controls relavite text height position.
        text  along path,
        text = {#7},
        text align = center,        
    },
    decorate
    ]
    (#5:#4) arc (#5:#6:#4);
}

% arcarrow, this is mine, for beerware purpose...
% Function: Draw an arrow from arctex coordinate specific nodes to another 
% Arrow start at the start of arctext box and could be shifted to change the position
% to avoid go over another box.
% Var: 1:Start coordinate 2:End coordinate 3:angle to shift from acrtext box  
\def\arcarrow(#1)(#2)[#3]{
    \draw[thick,->,>=latex] 
        let \p1 = (#1), \p2 = (#2), % To access cartesian coordinates x, and y.
            \n1 = {veclen(\x1,\y1)}, % Distance from the origin
            \n2 = {veclen(\x2,\y2)}, % Distance from the origin
            \n3 = {atan2(\y1,\x1)} % Angle where acrtext starts.
        in (\n3-#3: \n1) -- (\n3-#3: \n2); % Draw the arrow.
}


\begin{document}
\begin{tikzpicture}[
	    % Environment Cfg
	    font=\sf    \scriptsize,
	    % Styles
	    myarrow/.style={
	        thick,
	        -latex,
	    },
	    Center/.style ={
	        circle,
	        fill=ocre,
	        text=white,
	        align=center,
	        font =\footnotesize\bf,
	        inner sep=1pt,          
	    },
	    RedArc/.style ={
	        color=black,
	        thick,
	        fill=ocre,
	        blur shadow, 
	    },
	    SkyArc/.style ={
	        color=skybox,
	        thick,
	        fill=sky,
	        blur shadow, 
	    },
    ]

    % Drawing the center for AI
    \node[Center](AI) at (0,0) { Artificial \\ Intelligence \\(AI)\\ Agent};
    \coordinate (AROUND) at (0:1.2); 

    % Drawing the Text Arcs
    % Format: \Arctext[ID][box-style][box-height](radious)(start-angl)(end-angl){|text-styles| Text}
    
	% Machine Learning
    \arctext[ML][RedArc][8pt](2.75)(140)(75){|\footnotesize\bf\color{white}| Machine Learning};
    \arctext[REIN][SkyArc][5pt](3.50)(140)(75){|\scriptsize\color{black}| Reinforcement Learning};
    \arctext[KNOW][SkyArc][5pt](4.00)(110)(75){|\scriptsize| Knowledge L.};
    \arctext[SUPL][SkyArc][5pt](4.00)(140)(115){|\scriptsize| Supervised L.};
    \arctext[PROB][SkyArc][5pt](4.50)(140)(75){|\scriptsize\color{black}| Probabilistic Models};
    
    % Problem Solving
    \arctext[SOLV][RedArc][8pt](2.75)(70)(23){|\footnotesize\bf\color{white}| Problem Solving};
    \arctext[SRCH][SkyArc][5pt](3.50)(70)(23){|\scriptsize\color{black}| Search };
    \arctext[HEUR][SkyArc][5pt](4.00)(70)(50){|\scriptsize| Heuristics};
	\arctext[ADVS][SkyArc][5pt](4.00)(47)(23){|\scriptsize| Adversarial S.};    
	\arctext[CONST][SkyArc][5pt](4.50)(70)(23){|\scriptsize\color{black}| Constraint Analysis };
	
	% Natural Language Processing
    \arctext[NLP][RedArc][8pt](2.75)(15)(-15){|\footnotesize\bf\color{white}| NLP};
    \arctext[TRAN][SkyArc][5pt](3.50)(20)(-20){|\scriptsize\color{black}| Machine translation};
    \arctext[SPER][SkyArc][5pt](4.00)(20)(-20){|\scriptsize| Speech Recognition};
    \arctext[INFX][SkyArc][5pt](4.50)(20)(-20){|\scriptsize\color{black}| Information Extraction};

    % Decision Making
    \arctext[DEC][RedArc][8pt](2.75)(290)(340){|\footnotesize\bf\color{white}| Decision Making};
    \arctext[LOG][SkyArc][5pt](3.50)(290)(335){|\scriptsize\color{black}| Logic};
    \arctext[KNOW][SkyArc][5pt](4.00)(290)(335){|\scriptsize| Knowledge Engineering};
    \arctext[PLAN][SkyArc][5pt](4.50)(290)(335){|\scriptsize| Real World Planning};
    
    % Reasoning
    \arctext[REAS][RedArc][8pt](2.75)(250)(285){|\footnotesize\bf\color{white}|  Reasoning};
    \arctext[PROB][SkyArc][5pt](3.50)(250)(285){|\scriptsize\color{black}| Probabilistic R.};
    \arctext[REAT][SkyArc][5pt](4.00)(250)(285){|\scriptsize| Reasoning over time};
    \arctext[UNQU][SkyArc][5pt](4.50)(248)(287){|\scriptsize\color{black}| Uncertainty Quantification};
    
    % Robotics
    \arctext[ROB][RedArc][8pt](2.75)(210)(245){|\footnotesize\bf\color{white}|  Robotics};
    \arctext[RPER][SkyArc][5pt](3.50)(210)(245){|\scriptsize\color{black}| R. Perception};
    \arctext[RACT][SkyArc][5pt](4.00)(210)(245){|\scriptsize| R. Actuation};
    \arctext[MAPL][SkyArc][5pt](4.50)(210)(245){|\scriptsize\color{black}| Mapping, Localization};
    
    % Object Recognition
    \arctext[RECO][RedArc][8pt](2.75)(145)(205){|\footnotesize\bf\color{white}| Object Recognition};
    \arctext[VIS][SkyArc][5pt](3.50)(145)(205){|\scriptsize\color{black}| Vision};
    \arctext[IMGP][SkyArc][5pt](4.00)(145)(205){|\scriptsize\color{black}| Image Processing };
    \arctext[MOTC][SkyArc][5pt](4.50)(145)(205){|\scriptsize\color{black}| Motion, shading, contour analysis};   
  

    %ADITIONAL EXTERNAL ARC
%    \arctext[NEW][
%        color=white,
%        shade,      
%        upper left=blue,
%        upper right=black!50,
%        lower left=blue,
%        lower right=blue!50,
%        rounded corners = 8pt
%        ][8pt](5.2)(270)(-90){|\footnotesize\bf\color{white}| "An [artificial intelligence] agent is anything that can be viewed as perceiving its environment through sensors and acting upon that environment through actuators." Norvig and Russell};

    % Drawing the Arrows from contributing branch to AI
    % Format: \arcarrow(above/below ID)(abobe/below ID)[shift]
    \arcarrow(below ML)(AROUND)[30];
    \arcarrow(below SOLV)(AROUND)[24];
    \arcarrow(below NLP)(AROUND)[15];
    \arcarrow(below DEC)(AROUND)[-25];
    \arcarrow(below REAS)(AROUND)[-17];
    \arcarrow(below ROB)(AROUND)[-19];
    \arcarrow(below RECO)(AROUND)[-33];

    % Same level Arrows. Not needed now
    % \draw[myarrow] (left SSNX) -- (right DUAM);
    % \draw[myarrow] (left ML) -- (left SRel);
    % \draw[myarrow] (left SCap) -- (right ML);

	% Legend
    \draw[myarrow] (-5,-5) coordinate (legend) -- ++(.8,0) node[anchor=west] {(contribution)};
    \draw[RedArc] (legend)++(0,-0.4) rectangle ++(.8,-.3)++(0,.2) node[anchor=west] {disciplines};
    \draw[SkyArc] (legend)++(0,-1) rectangle ++(.8,-.3)++(0,.2) node[anchor=west, color=black] {requirements};
        
		% \node[draw] at (0,0) {};
		% \node[draw,align=left] at (3,0) {};
		\node[text width=3cm] at (3,-6.25) {\begin{tiny}Copyright \textcopyright Alfonso R. Reyes, 2020\end{tiny}};

\end{tikzpicture}  
\end{document}